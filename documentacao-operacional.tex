\documentclass[12pt, a4paper]{article}

% --- Pacotes Mínimos ---
\usepackage[utf8]{inputenc}
\usepackage[T1]{fontenc}
\usepackage[portuguese]{babel}
\usepackage[margin=1in]{geometry}
\usepackage{hyperref}
\usepackage{xcolor}
\usepackage{graphicx}
\usepackage{booktabs}
\usepackage{enumitem}

% --- Cores e Comandos Simplificados (WCAG Contraste Refinado) ---
\definecolor{fisiopetBlue}{HTML}{0070C0}
\definecolor{fisiopetRed}{HTML}{C00000}
\definecolor{noteBG}{HTML}{FFFFFF}
\definecolor{noteBorder}{HTML}{C00000}
\definecolor{successGreen}{HTML}{22C55E}
\definecolor{warningOrange}{HTML}{F59E0B}

\newcommand{\NoteBox}[1]{%
    \vspace{4pt}
    \noindent\fcolorbox{noteBorder}{noteBG}{\parbox{\dimexpr\linewidth-2\fboxsep-2\fboxrule\relax}{\textbf{\textcolor{fisiopetRed}{ATENÇÃO:}} #1}}%
    \vspace{4pt}
}

\newcommand{\SuccessBox}[1]{%
    \vspace{4pt}
    \noindent\fcolorbox{successGreen}{noteBG}{\parbox{\dimexpr\linewidth-2\fboxsep-2\fboxrule\relax}{\textbf{\textcolor{successGreen}{✓ DICA:}} #1}}%
    \vspace{4pt}
}

\setlength{\parskip}{0.5em}
\setlength{\parindent}{0pt}

\hypersetup{
    colorlinks=true,
    linkcolor=fisiopetBlue,
    urlcolor=fisiopetBlue,
    citecolor=fisiopetBlue
}

\begin{document}

% --- LOGO E TÍTULO ---
\begin{center}
    \includegraphics[width=0.45\linewidth]{logo.png}
    \vspace{8pt}
    \\
    \textbf{\textcolor{fisiopetBlue}{\Large DOCUMENTAÇÃO OPERACIONAL}}
    \vspace{4pt}
    \\
    \textcolor{fisiopetBlue}{\textit{Sistema de Fila de Atendimento Veterinário}}
\end{center}
\vspace{15pt}

O sistema digital de fila de atendimento foi desenvolvido para o Hospital Veterinário \textbf{\textcolor{fisiopetBlue}{Fisiopet}}. Ele permite que recepcionistas adicionem pacientes, veterinários realizem o atendimento e exibe a fila em tempo real na sala de espera.

% --- SEÇÃO 1: O que o Sistema Faz ---
\section*{\Large\textbf{\textcolor{fisiopetBlue}{1. O Que o Sistema Faz}}}

O sistema VetQueue é uma solução completa para gestão de filas e atendimentos veterinários, oferecendo funcionalidades específicas para cada tipo de usuário.

\subsection*{\textbf{Para Recepcionistas}}

Recepcionistas têm acesso completo ao sistema e podem:

\begin{itemize}[leftmargin=*]
    \item \textbf{Adicionar pacientes à fila} - Incluir novos atendimentos rapidamente
    \item \textbf{Chamar o próximo paciente} - Gerenciar o fluxo da fila
    \item \textbf{Cancelar entradas} - Remover pacientes da fila quando necessário
    \item \textbf{Visualizar histórico completo} - Consultar todos os atendimentos realizados
    \item \textbf{Acessar relatórios detalhados} - Estatísticas e métricas de atendimento
    \item \textbf{Gerenciar pacientes} - Cadastrar e editar informações de pacientes e tutores
    \item \textbf{Administrar o sistema} - Gerenciar usuários, salas e serviços
    \item \textbf{Visualizar auditoria} - Acompanhar todas as ações realizadas no sistema
\end{itemize}

\subsection*{\textbf{Para Veterinários}}

Veterinários têm acesso às funcionalidades de atendimento:

\begin{itemize}[leftmargin=*]
    \item \textbf{Fazer check-in em uma sala} - Selecionar o consultório de trabalho
    \item \textbf{Chamar pacientes} - Chamar o próximo da fila ou pacientes específicos
    \item \textbf{Iniciar e finalizar atendimentos} - Controlar o status do atendimento
    \item \textbf{Registrar consultas e vacinações} - Criar prontuário eletrônico completo
    \item \textbf{Visualizar prontuário do paciente} - Acessar histórico médico completo
    \item \textbf{Acompanhar fila em tempo real} - Ver pacientes aguardando e em atendimento
\end{itemize}

\subsection*{\textbf{Para o Display na TV}}

O sistema oferece uma página especial para exibição pública:

\begin{itemize}[leftmargin=*]
    \item \textbf{Exibe a fila atual em tempo real} - Mostra pacientes aguardando
    \item \textbf{Apresenta paciente em atendimento} - Indica qual animal está sendo atendido
    \item \textbf{Atualização automática} - A página atualiza sozinha sem necessidade de recarregar
    \item \textbf{Interface otimizada para TV} - Design pensado para visualização à distância
    \item \textbf{Indicadores visuais claros} - Cores e ícones para fácil identificação
\end{itemize}

% --- SEÇÃO 2: Acesso ---
\section*{\Large\textbf{\textcolor{fisiopetBlue}{2. Como Acessar o Sistema}}}

\subsection*{\textbf{Acesso de Usuários (Recepção e Consultórios)}}

Para acessar o sistema, siga os seguintes passos:

\begin{enumerate}[leftmargin=*]
    \item Abra o navegador web (Chrome, Edge ou Firefox são recomendados)
    \item Digite o endereço: \textbf{\url{https://fisiopet.petshopcisnebranco.com.br}}
    \item Aguarde o carregamento da página de login
    \item Digite suas credenciais (usuário e senha)
    \item Clique em "Entrar" ou pressione Enter
\end{enumerate}

\SuccessBox{O sistema é compatível com dispositivos móveis (tablets e smartphones), permitindo acesso flexível em qualquer lugar da clínica.}

\subsection*{\textbf{Credenciais de Acesso}}

\textbf{IMPORTANTE:} As credenciais abaixo são para uso inicial. Após o primeiro acesso, você deve alterar sua senha.

\begin{center}
\begin{tabular}{|p{0.3\linewidth}|p{0.25\linewidth}|p{0.35\linewidth}|}
    \toprule
    \textbf{Função} & \textbf{Usuário} & \textbf{Senha} \\
    \midrule
    Recepcionista & \texttt{recepcao} & \texttt{senha123} \\
    Veterinário & \texttt{drjoao} & \texttt{senha123} \\
    \bottomrule
\end{tabular}
\end{center}

\NoteBox{Após o primeiro acesso, altere sua senha nas configurações do sistema. A senha padrão é temporária e deve ser trocada por segurança.}

\subsection*{\textbf{Troubleshooting de Login}}

Se você encontrar problemas ao fazer login:

\begin{itemize}[leftmargin=*]
    \item \textbf{Verifique a conexão com a internet} - Certifique-se de que está conectado
    \item \textbf{Confirme usuário e senha} - Verifique se não há espaços antes ou depois
    \item \textbf{Limpe o cache do navegador} - Às vezes cache antigo pode causar problemas
    \item \textbf{Tente outro navegador} - Chrome, Edge ou Firefox são recomendados
    \item \textbf{Verifique se o sistema está online} - Entre em contato com o administrador se o problema persistir
\end{itemize}

% --- SEÇÃO 3: Display na TV ---
\section*{\Large\textcolor{fisiopetBlue}{3. Como Configurar o Display na TV}}

O sistema utiliza uma página especial para a televisão da sala de espera, permitindo que os tutores acompanhem o status de atendimento em tempo real.

\subsection*{\textbf{Configuração Inicial}}

\begin{enumerate}[leftmargin=*]
    \item \textbf{Conecte a TV ao computador} - Via cabo HDMI ou rede (se a TV for smart)
    \item \textbf{Ligue o computador e a TV}
    \item \textbf{Acesse a página de display}: Digite no navegador: \textbf{\url{https://fisiopet.petshopcisnebranco.com.br/display}}
    \item \textbf{Ative o modo tela cheia}: Pressione \textbf{F11} no teclado para entrar em modo tela cheia
    \item \textbf{Configure o navegador} - Desative proteção de tela e suspensão automática
\end{enumerate}

\subsection*{\textbf{Configurações Recomendadas}}

Para garantir o melhor funcionamento do display:

\begin{itemize}[leftmargin=*]
    \item \textbf{Modo tela cheia permanente} - Use F11 para ativar
    \item \textbf{Desativar proteção de tela} - Configure o sistema operacional para não ativar proteção de tela
    \item \textbf{Abrir automaticamente ao iniciar} - Configure o navegador para abrir a página ao iniciar o Windows
    \item \textbf{Atualização automática} - A página atualiza automaticamente a cada 3 segundos
    \item \textbf{Som opcional} - O sistema pode emitir alertas sonoros quando um paciente é chamado
\end{itemize}

\SuccessBox{Se o display não estiver atualizando, pressione F5 para recarregar a página manualmente. O sistema funciona melhor quando o computador está sempre ligado e conectado à internet.}

% --- SEÇÃO 4: Guia de Uso Rápido ---
\section*{\Large\textcolor{fisiopetBlue}{4. Guia de Uso Rápido}}

\subsection*{\textbf{Para Recepcionistas}}

\subsubsection*{\textbf{Adicionar Paciente à Fila}}

\begin{enumerate}[leftmargin=*]
    \item Na página principal, clique no botão \textbf{"Adicionar"} (canto superior direito)
    \item Preencha os dados do paciente:
    \begin{itemize}
        \item \textbf{Tutor} - Nome do responsável (pode buscar em pacientes existentes)
        \item \textbf{Pet} - Nome do animal (pode buscar ou digitar novo)
        \item \textbf{Tipo de Serviço} - Selecione: Consulta, Vacinação, Cirurgia, Exame ou Banho e Tosa
        \item \textbf{Prioridade} - Normal, Alta ou Emergência
        \item \textbf{Horário Agendado} (opcional) - Se o paciente tem horário marcado
    \end{itemize}
    \item Clique em \textbf{"Adicionar à Fila"}
    \item O paciente aparecerá imediatamente na fila e no display da TV
\end{enumerate}

\SuccessBox{\textbf{Atalho de Teclado:} Pressione \textbf{Ctrl+N} para abrir rapidamente o formulário de adicionar paciente (apenas para recepcionistas).}

\subsubsection*{\textbf{Chamar Próximo Paciente}}

\begin{enumerate}[leftmargin=*]
    \item Clique no botão \textbf{"Chamar Próximo"} (canto superior da página)
    \item Se houver veterinário com check-in, selecione a sala
    \item O sistema chamará automaticamente o próximo paciente da fila
    \item O nome aparecerá no display da TV
    \item O paciente será movido para status "Chamado"
\end{enumerate}

\SuccessBox{\textbf{Atalho de Teclado:} Pressione \textbf{Enter} para chamar o próximo paciente quando houver pacientes aguardando (sem estar digitando em um campo).}

\subsubsection*{\textbf{Visualizar Histórico}}

\begin{enumerate}[leftmargin=*]
    \item Na página da fila, clique na aba \textbf{"Histórico"}
    \item Selecione o período desejado (data inicial e final)
    \item Use os filtros para buscar por:
    \begin{itemize}
        \item Nome do tutor
        \item Nome do pet
        \item Tipo de serviço
    \end{itemize}
    \item Visualize todos os atendimentos concluídos no período
\end{enumerate}

\subsubsection*{\textbf{Ver Relatórios}}

\begin{enumerate}[leftmargin=*]
    \item Na página da fila, clique na aba \textbf{"Relatórios"}
    \item Selecione o período desejado
    \item Visualize:
    \begin{itemize}
        \item Total de atendimentos
        \item Tempo médio de espera
        \item Tempo médio de atendimento
        \item Taxa de cancelamento
        \item Distribuição por tipo de serviço
        \item Distribuição por prioridade
        \item Top veterinários
        \item Horários de pico
        \item Utilização das salas
    \end{itemize}
\end{enumerate}

\subsubsection*{\textbf{Cancelar Entrada da Fila}}

\begin{enumerate}[leftmargin=*]
    \item Localize o paciente na fila
    \item Clique no botão \textbf{"Cancelar"} (ícone de X)
    \item Confirme a ação na caixa de diálogo
    \item O paciente será removido da fila e marcado como cancelado
\end{enumerate}

\subsection*{\textbf{Para Veterinários}}

\subsubsection*{\textbf{Check-in na Sala}}

\begin{enumerate}[leftmargin=*]
    \item Ao fazer login, o sistema pode solicitar que você selecione uma sala
    \item Selecione o consultório onde você vai trabalhar
    \item Clique em \textbf{"Confirmar"}
    \item Você pode trocar de sala a qualquer momento clicando no seletor de sala no topo da página
\end{enumerate}

\NoteBox{\textbf{IMPORTANTE:} Você precisa estar com check-in em uma sala para poder chamar pacientes. O sistema só permite chamar pacientes para salas que têm veterinário com check-in.}

\subsubsection*{\textbf{Chamar Próximo Paciente}}

\begin{enumerate}[leftmargin=*]
    \item Certifique-se de estar com check-in em uma sala
    \item Clique no botão \textbf{"Chamar Próximo"} (canto superior da página)
    \item O sistema chamará automaticamente o próximo paciente da fila para sua sala
    \item O paciente aparecerá no display da TV
    \item O status mudará para "Chamado"
\end{enumerate}

\SuccessBox{\textbf{Atalho:} Pressione \textbf{Enter} para chamar o próximo paciente (quando não estiver digitando em um campo).}

\subsubsection*{\textbf{Chamar Paciente Específico}}

\begin{enumerate}[leftmargin=*]
    \item Localize o paciente desejado na fila
    \item Clique no botão \textbf{"Chamar"} (ícone de telefone) no card do paciente
    \item O paciente será chamado para sua sala
    \item O status mudará para "Chamado"
\end{enumerate}

\subsubsection*{\textbf{Iniciar Atendimento}}

\begin{enumerate}[leftmargin=*]
    \item Após chamar o paciente, localize-o na lista (status "Chamado")
    \item Clique no botão \textbf{"Iniciar Atendimento"}
    \item O status mudará para "Em Atendimento"
    \item O tempo de atendimento começará a ser contabilizado
\end{enumerate}

\subsubsection*{\textbf{Registrar Consulta ou Vacinação}}

\begin{enumerate}[leftmargin=*]
    \item Durante o atendimento, localize o paciente na fila (status "Em Atendimento")
    \item Clique no botão \textbf{"Ver Prontuário"} (ícone de pasta/arquivo)
    \item Na janela que abrir, selecione a aba \textbf{"Consultas"} ou \textbf{"Vacinas"}
    \item Clique em \textbf{"Nova Consulta"} ou \textbf{"Nova Vacina"}
    \item Preencha o formulário:
    \begin{itemize}
        \item \textbf{Consulta:} Data, Diagnóstico, Tratamento, Prescrição, Peso (kg), Observações
        \item \textbf{Vacina:} Nome da vacina, Data de aplicação, Lote, Próxima dose, Observações
    \end{itemize}
    \item Clique em \textbf{"Salvar"}
    \item O registro será salvo no prontuário eletrônico do paciente
\end{enumerate}

\SuccessBox{Se o paciente já tem um prontuário cadastrado, o sistema carregará automaticamente as informações. Caso contrário, você pode criar um novo paciente durante o registro.}

\subsubsection*{\textbf{Finalizar Atendimento}}

\begin{enumerate}[leftmargin=*]
    \item Localize o paciente em atendimento (status "Em Atendimento")
    \item Certifique-se de ter registrado consulta/vacinação se necessário
    \item Clique no botão \textbf{"Finalizar Atendimento"}
    \item O status mudará para "Finalizado"
    \item O atendimento será movido para o histórico
    \item O tempo total de atendimento será registrado
\end{enumerate}

% --- SEÇÃO 5: Funcionalidades Disponíveis ---
\section*{\Large\textcolor{fisiopetBlue}{5. Funcionalidades Disponíveis e Status}}

O sistema possui as seguintes funcionalidades implementadas e ativas:

\begin{itemize}[leftmargin=*]
    \item $\checkmark$ \textbf{Adicionar paciente à fila} - Com ou sem cadastro prévio
    \item $\checkmark$ \textbf{Chamar próximo da fila} - Com exibição automática na TV
    \item $\checkmark$ \textbf{Chamar paciente específico} - Para atendimento prioritário
    \item $\checkmark$ \textbf{Iniciar e finalizar atendimento} - Controle completo do fluxo
    \item $\checkmark$ \textbf{Registrar consultas} - Prontuário eletrônico completo
    \item $\checkmark$ \textbf{Registrar vacinações} - Com controle de próximas doses
    \item $\checkmark$ \textbf{Visualizar histórico de atendimentos} - Com filtros avançados
    \item $\checkmark$ \textbf{Cancelar entrada da fila} - Com registro de motivo
    \item $\checkmark$ \textbf{Relatórios detalhados} - Estatísticas e métricas completas
    \item $\checkmark$ \textbf{Gestão de pacientes} - Cadastro completo de pacientes e tutores
    \item $\checkmark$ \textbf{Prontuário eletrônico} - Histórico médico completo
    \item $\checkmark$ \textbf{Check-in em salas} - Gerenciamento de consultórios
    \item $\checkmark$ \textbf{Atribuição de veterinário} - Recepção pode atribuir pacientes a veterinários específicos
    \item $\checkmark$ \textbf{Agendamentos} - Suporte a horários agendados com prioridade
    \item $\checkmark$ \textbf{Auditoria} - Log completo de todas as ações (apenas recepcionistas)
    \item $\checkmark$ \textbf{Administração} - Gestão de usuários, salas e serviços (apenas recepcionistas)
\end{itemize}

% --- SEÇÃO 6: Prontuário Eletrônico ---
\section*{\Large\textcolor{fisiopetBlue}{6. Prontuário Eletrônico}}

O sistema possui um módulo completo de prontuário eletrônico que permite registrar e consultar todo o histórico médico do paciente.

\subsection*{\textbf{O Que é o Prontuário Eletrônico}}

O prontuário eletrônico é um registro digital completo de todas as informações médicas do animal, incluindo:

\begin{itemize}[leftmargin=*]
    \item \textbf{Dados do paciente} - Nome, espécie, raça, idade, peso atual
    \item \textbf{Dados do tutor} - Nome, telefone, e-mail, endereço
    \item \textbf{Histórico de consultas} - Todas as consultas realizadas com diagnósticos e tratamentos
    \item \textbf{Histórico de vacinações} - Vacinas aplicadas e próximas doses
    \item \textbf{Informações importantes} - Alergias, medicações em uso, condições especiais
    \item \textbf{Histórico de atendimentos} - Todos os atendimentos na fila relacionados ao paciente
\end{itemize}

\subsection*{\textbf{Como Acessar o Prontuário}}

\begin{enumerate}[leftmargin=*]
    \item \textbf{Durante o atendimento:} Clique em "Ver Prontuário" na entrada da fila
    \item \textbf{Pela página de pacientes:} Acesse "Pacientes" no menu principal e selecione o paciente desejado
    \item \textbf{Pelo histórico:} Ao visualizar o histórico, clique em "Ver Prontuário" em qualquer atendimento
\end{enumerate}

\subsection*{\textbf{Como Registrar uma Consulta}}

\begin{enumerate}[leftmargin=*]
    \item Acesse o prontuário do paciente (métodos acima)
    \item Selecione a aba \textbf{"Consultas"}
    \item Clique em \textbf{"Nova Consulta"}
    \item Preencha os campos:
    \begin{itemize}
        \item \textbf{Data} - Data e hora da consulta (padrão: data/hora atual)
        \item \textbf{Diagnóstico} - Diagnóstico principal
        \item \textbf{Tratamento} - Tratamento prescrito
        \item \textbf{Prescrição} - Medicações e dosagens
        \item \textbf{Peso (kg)} - Peso atual do animal
        \item \textbf{Observações} - Notas adicionais
    \end{itemize}
    \item Clique em \textbf{"Salvar Consulta"}
    \item A consulta será registrada no histórico do paciente
\end{enumerate}

\subsection*{\textbf{Como Registrar uma Vacinação}}

\begin{enumerate}[leftmargin=*]
    \item Acesse o prontuário do paciente
    \item Selecione a aba \textbf{"Vacinas"}
    \item Clique em \textbf{"Nova Vacina"}
    \item Preencha os campos:
    \begin{itemize}
        \item \textbf{Nome da Vacina} - Ex: V8, V10, Antirrábica, etc.
        \item \textbf{Data de Aplicação} - Data em que foi aplicada
        \item \textbf{Lote} - Número do lote da vacina (opcional)
        \item \textbf{Próxima Dose} - Data para próxima aplicação (se aplicável)
        \item \textbf{Observações} - Notas adicionais
    \end{itemize}
    \item Clique em \textbf{"Salvar Vacina"}
    \item A vacinação será registrada e o sistema alertará sobre próximas doses
\end{enumerate}

\subsection*{\textbf{Benefícios do Prontuário Eletrônico}}

\begin{itemize}[leftmargin=*]
    \item \textbf{Histórico completo} - Acesso rápido a todo o histórico médico
    \item \textbf{Rastreamento de vacinações} - Controle de vacinas aplicadas e próximas doses
    \item \textbf{Diagnósticos e tratamentos} - Registro de todas as consultas e tratamentos
    \item \textbf{Evolução do peso} - Acompanhamento do peso ao longo do tempo
    \item \textbf{Informações críticas} - Alergias e medicações sempre visíveis
    \item \textbf{Acesso rápido} - Disponível durante qualquer atendimento
\end{itemize}

% --- SEÇÃO 7: Funcionalidades Avançadas (Recepcionistas) ---
\section*{\Large\textcolor{fisiopetBlue}{7. Funcionalidades Avançadas (Apenas Recepcionistas)}}

Recepcionistas têm acesso a funcionalidades administrativas adicionais para gerenciar o sistema.

\subsection*{\textbf{Gestão de Pacientes}}

Acesse pelo menu "Pacientes" na página principal:

\begin{itemize}[leftmargin=*]
    \item \textbf{Cadastrar novo paciente} - Incluir paciente e tutor no sistema
    \item \textbf{Editar informações} - Atualizar dados de pacientes existentes
    \item \textbf{Buscar pacientes} - Localizar rapidamente por nome, tutor ou espécie
    \item \textbf{Visualizar prontuário completo} - Acessar todo o histórico médico
    \item \textbf{Informações importantes} - Cadastrar alergias, medicações e condições especiais
\end{itemize}

\subsection*{\textbf{Administração do Sistema}}

Acesse pelo menu "Admin" no canto superior direito:

\subsubsection*{\textbf{Gestão de Usuários}}

\begin{itemize}[leftmargin=*]
    \item Criar novos usuários (recepcionistas e veterinários)
    \item Editar informações de usuários existentes
    \item Alterar senhas
    \item Desativar usuários (quando necessário)
\end{itemize}

\subsubsection*{\textbf{Gestão de Salas}}

\begin{itemize}[leftmargin=*]
    \item Adicionar novas salas/consultórios
    \item Editar nomes de salas
    \item Ativar/desativar salas
    \item Visualizar salas existentes
\end{itemize}

\subsubsection*{\textbf{Gestão de Serviços}}

\begin{itemize}[leftmargin=*]
    \item Adicionar novos tipos de serviço
    \item Editar serviços existentes
    \item Ativar/desativar serviços
    \item Gerenciar tipos de atendimento disponíveis
\end{itemize}

\subsection*{\textbf{Auditoria}}

A aba "Auditoria" na página da fila permite:

\begin{itemize}[leftmargin=*]
    \item Visualizar log completo de todas as ações no sistema
    \item Filtrar por usuário, data, ação ou tipo de entidade
    \item Rastrear todas as alterações realizadas
    \item Identificar quem realizou cada ação
    \item Verificar histórico de modificações
\end{itemize}

\NoteBox{A auditoria é importante para segurança e rastreabilidade. Todas as ações importantes são registradas automaticamente no sistema.}

% --- SEÇÃO 8: Relatórios Detalhados ---
\section*{\Large\textcolor{fisiopetBlue}{8. Relatórios e Estatísticas}}

O sistema oferece relatórios detalhados para análise de desempenho e gestão.

\subsection*{\textbf{Relatórios Disponíveis}}

\subsubsection*{\textbf{Métricas Gerais}}

\begin{itemize}[leftmargin=*]
    \item \textbf{Total de Atendimentos} - Quantidade total no período
    \item \textbf{Média por Dia} - Média de atendimentos diários
    \item \textbf{Tempo Médio de Espera} - Tempo médio que pacientes aguardam
    \item \textbf{Tempo Médio de Atendimento} - Duração média dos atendimentos
    \item \textbf{Taxa de Cancelamento} - Percentual de atendimentos cancelados
\end{itemize}

\subsubsection*{\textbf{Análises Gráficas}}

\begin{itemize}[leftmargin=*]
    \item \textbf{Distribuição por Tipo de Serviço} - Gráfico de barras mostrando quantidade por serviço
    \item \textbf{Distribuição por Prioridade} - Gráfico de pizza com emergências, alta e normal
    \item \textbf{Top Veterinários} - Gráfico de barras horizontais com mais atendimentos
    \item \textbf{Horários de Pico} - Gráfico de área mostrando horários mais movimentados
    \item \textbf{Utilização das Salas} - Barras de progresso e estatísticas por consultório
\end{itemize}

\subsection*{\textbf{Como Usar os Relatórios}}

\begin{enumerate}[leftmargin=*]
    \item Acesse a aba \textbf{"Relatórios"} na página da fila
    \item Selecione o \textbf{período desejado} (data inicial e final)
    \item Os relatórios serão atualizados automaticamente
    \item Visualize os gráficos e métricas
    \item Use os dados para análise e tomada de decisão
\end{enumerate}

\SuccessBox{Os relatórios são atualizados em tempo real. Você pode alterar o período a qualquer momento para análise de diferentes intervalos.}

% --- SEÇÃO 9: Solução de Problemas ---
\section*{\Large\textcolor{fisiopetBlue}{9. Suporte Técnico e Solução de Problemas}}

\subsection*{\textbf{Problemas Comuns e Soluções}}

\begin{center}
\begin{tabular}{p{0.28\linewidth} p{0.68\linewidth}}
    \toprule
    \textbf{Problema} & \textbf{Solução} \\
    \midrule
    Não consigo fazer login & 
    \begin{enumerate}[leftmargin=*, nosep, topsep=0pt]
        \item Verifique usuário e senha (sem espaços)
        \item Confirme se o sistema está online
        \item Limpe o cache do navegador
        \item Tente outro navegador
        \item Contate o administrador
    \end{enumerate} \\
    \midrule
    A fila não aparece na TV & 
    \begin{enumerate}[leftmargin=*, nosep, topsep=0pt]
        \item Verifique o endereço: \texttt{/display}
        \item Recarregue a página (F5)
        \item Confirme a conexão com internet
        \item Verifique se há pacientes na fila
        \item Certifique-se de estar em modo tela cheia (F11)
    \end{enumerate} \\
    \midrule
    Não consigo chamar o próximo & 
    \begin{enumerate}[leftmargin=*, nosep, topsep=0pt]
        \item Verifique se há pacientes aguardando
        \item Veterinários: confirme check-in na sala
        \item Recepcionistas: verifique se há veterinário com check-in
        \item Recarregue a página (F5)
        \item Verifique conexão com internet
    \end{enumerate} \\
    \midrule
    O sistema está lento & 
    \begin{enumerate}[leftmargin=*, nosep, topsep=0pt]
        \item Verifique a conexão com internet
        \item Feche abas desnecessárias do navegador
        \item Limpe o cache do navegador
        \item Reinicie o navegador
        \item Se persistir, acione o suporte técnico
    \end{enumerate} \\
    \midrule
    Não consigo registrar consulta/vacinação & 
    \begin{enumerate}[leftmargin=*, nosep, topsep=0pt]
        \item Verifique se o paciente tem prontuário
        \item Confirme que está durante um atendimento
        \item Clique em "Ver Prontuário" na entrada da fila
        \item Selecione a aba correta (Consultas ou Vacinas)
        \item Preencha todos os campos obrigatórios
    \end{enumerate} \\
    \midrule
    A página não atualiza sozinha & 
    \begin{enumerate}[leftmargin=*, nosep, topsep=0pt]
        \item Recarregue manualmente (F5)
        \item Verifique conexão com internet
        \item Limpe o cache do navegador
        \item Verifique se o JavaScript está habilitado
        \item Tente outro navegador
    \end{enumerate} \\
    \midrule
    Erro ao salvar dados & 
    \begin{enumerate}[leftmargin=*, nosep, topsep=0pt]
        \item Verifique conexão com internet
        \item Certifique-se de preencher campos obrigatórios
        \item Verifique se não há caracteres especiais problemáticos
        \item Tente novamente após alguns segundos
        \item Se persistir, contate o suporte
    \end{enumerate} \\
    \bottomrule
\end{tabular}
\end{center}

\subsection*{\textbf{Dicas de Performance}}

\begin{itemize}[leftmargin=*]
    \item \textbf{Mantenha poucas abas abertas} - Muitas abas podem deixar o navegador lento
    \item \textbf{Use navegadores atualizados} - Versões antigas podem ter problemas de performance
    \item \textbf{Conecte-se via Wi-Fi estável} - Conexão instável pode causar lentidão
    \item \textbf{Reinicie o navegador periodicamente} - Especialmente após uso prolongado
    \item \textbf{Limpe o cache regularmente} - Cache excessivo pode causar problemas
\end{itemize}

\subsection*{\textbf{Canais de Suporte}}

\subsubsection*{\textbf{Contato Principal (Recomendado)}}

\begin{itemize}[leftmargin=*]
    \item \textbf{Vinícius - Administrador do Sistema}
    \item \textbf{WhatsApp:} \href{https://wa.me/5513974023753}{(13) 97402-3753}
    \item \textbf{Disponibilidade:} Horário comercial e emergências técnicas
\end{itemize}

\subsubsection*{\textbf{Contato Alternativo}}

\begin{itemize}[leftmargin=*]
    \item \textbf{Fernando} - Pode ajudar em caso de necessidade
    \item \textbf{Quando usar:} Se Vinícius não estiver disponível
\end{itemize}

\NoteBox{É recomendado entrar em contato direto com o Vinícius pelo WhatsApp para qualquer problema ou dúvida sobre o sistema. Ele tem acesso completo ao sistema e pode resolver a maioria dos problemas rapidamente.}

\subsection*{\textbf{Informações para Suporte}}

Ao entrar em contato com o suporte, tenha em mãos:

\begin{itemize}[leftmargin=*]
    \item \textbf{Descrição do problema} - O que está acontecendo
    \item \textbf{Mensagem de erro} - Se aparecer alguma mensagem, copie o texto
    \item \textbf{Navegador usado} - Chrome, Edge, Firefox, etc.
    \item \textbf{Usuário afetado} - Seu nome de usuário
    \item \textbf{Horário do problema} - Quando ocorreu
    \item \textbf{Passos para reproduzir} - O que você estava fazendo quando aconteceu
\end{itemize}

% --- SEÇÃO 10: Atalhos de Teclado ---
\section*{\Large\textcolor{fisiopetBlue}{10. Atalhos de Teclado e Dicas de Produtividade}}

O sistema possui atalhos de teclado para agilizar o trabalho diário.

\subsection*{\textbf{Atalhos Disponíveis}}

\begin{center}
\begin{tabular}{|p{0.25\linewidth}|p{0.70\linewidth}|}
    \toprule
    \textbf{Atalho} & \textbf{Ação} \\
    \midrule
    \texttt{Ctrl + N} & Abre formulário para adicionar paciente (apenas recepcionistas) \\
    \midrule
    \texttt{Enter} & Chama o próximo paciente da fila (quando não estiver digitando) \\
    \midrule
    \texttt{F11} & Ativa/desativa modo tela cheia (útil para display na TV) \\
    \midrule
    \texttt{F5} & Recarrega a página atual \\
    \bottomrule
\end{tabular}
\end{center}

\SuccessBox{Os atalhos funcionam apenas quando você não está digitando em um campo. Para usar Enter, certifique-se de que nenhum campo de texto está selecionado.}

\subsection*{\textbf{Dicas de Produtividade}}

\begin{itemize}[leftmargin=*]
    \item \textbf{Use o autocomplete} - Ao adicionar pacientes, o sistema sugere tutores e pets existentes
    \item \textbf{Aproveite os filtros} - No histórico, use filtros para encontrar rapidamente atendimentos
    \item \textbf{Atualização automática} - A fila atualiza sozinha a cada 3 segundos, não precisa recarregar
    \item \textbf{Prontuário durante atendimento} - Acesse o prontuário diretamente da fila sem navegar
    \item \textbf{Relatórios em tempo real} - Os relatórios são atualizados automaticamente ao mudar o período
\end{itemize}

% --- RODAPÉ ---
\vfill
\noindent\rule{\linewidth}{0.4pt}
\begin{flushleft}
\footnotesize
\textbf{Versão do Documento:} 2.0 \\
\textbf{Última atualização:} \today \\
\textbf{Sistema:} VetQueue - Sistema de Fila de Atendimento Veterinário \\
\textbf{Desenvolvido para:} Hospital Veterinário Fisiopet
\end{flushleft}

\end{document}



